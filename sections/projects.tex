%\vspace*{5pt}

\documentclass[a4paper, 12pt]{article}
\begin{document}

\subsection {{University of Arizona, Department of Astronomy \& Steward Observatory}}

\vspace*{5pt}
\subtext{Tool for Rapid Object Vetting and Examination (TROVE) \hfill Oct 2024 - Present}
\begin{zitemize}
    \item Actively developing a tool for connecting and vetting potential electromagnetic counterparts to non-localized multimessenger events, such as gravitational waves and neutrino observations.
\end{zitemize}

\vspace*{5pt}
\subtext{Searches After Gravitational waves Using ARizona Observatories (SAGUARO) \hfill May 2024 - Present}
\begin{zitemize}
    \item Maintain the SAGUARO Target and Observation Management (TOM) software infrastructure.
    \item Contribute features to the SAGUARO TOM including improved target vetting and user experience.
\end{zitemize}

\vspace*{5pt}
\subtext{The Open mulTiwavelength Transient Event Repository (OTTER) \hfill Aug 2023 - Present}
\begin{zitemize}
    \item Accumulate and clean $>100,000$ photometric observations of tidal disruption events from the literature into a customized JSON data schema stored as an ArangoDB document database.
    \item Develop the software API to access the cleaned dataset of photometry.
    \item Build a front-end web application for viewing, downloading, and contributing other datasets to the catalog.
\end{zitemize}

\vspace*{5pt}
\subtext{Radio Observations of Extreme Coronal Line Emitters \hfill Aug 2023 - Present}
\begin{zitemize}
    \item Reduce radio observations of Extreme Coronal Line Emitters using the standard CASA software.
    \item Analyze the results of the radio observations to better understand the connection between extreme coronal line emitters and tidal disruption events.
\end{zitemize}
\vspace*{8pt}

\subsection {{University of Hawaii at Manoa, Institute for Astronomy}}
\subtext{Research Intern - Tip of the Red Giant Branch Bounds on the NMDM Revisited \hfill Jun 2022 - Aug 2022}
\begin{zitemize}
\item Modified an open source stellar evolution simulation using Fortran.
\item Optimized a simulation by using python to train a deep neural network and use it as a simulation emulator.
\item Conducted a Bayesian statistical analysis, Markov Chain Monte Carlo, to constrain a particle physics property.
\item Code is available on GitHub and results will be presented in Franz et al. (2023), in progress (see second page).
\end{zitemize}

\vspace*{8pt}
\subsection {{University of California, Berkeley Search for Extraterrestrial Intelligence (SETI)}}
\subtext{Research Intern - Technosignature Search of Transiting TESS Targets of Interest \hfill Jun 2021 - May 2022}
\begin{zitemize}
\item Searched through and analyzed over 30 terabytes of Green Bank Telescope radio data for evidence of extraterrestrial intelligence using Python and Bash.
\item Optimized the existing search software by developing a a parallel processing algorithm using multiple compute nodes on a cluster.
\item Created visualizations of multi-dimensional radio signals using matplotlib.
\item Code is available on GitHub and results are published in Franz et al. (2022), \textit{Astronomical Journal}.
\end{zitemize}

\vspace*{8pt}

\subsection {{Siena College}}
\vspace*{5pt}

\subtext{{\texttt{hepfile} Development}\hfill May 2023 - July 2023}
\begin{zitemize}
    \item Developed the Python \texttt{hepfile} software to store so-called ``hetergeneous" datasets.
    \item Added tools for integration with existing Python software.
\end{zitemize}
\vspace*{5pt}

\subtext{{Senior Thesis}\hfill Sep 2022 - May 2023}
\begin{zitemize}
    \item Developed a pipeline to search Dark Energy Spectroscopic Instrument data for spectroscopic lenses.
    \item Analyzed spectroscopic lenses to extract source object properties.
\end{zitemize}
\vspace*{5pt}

\subtext{{Astrophysics Research Intern}\hfill Jan 2021 - Feb 2022}
\begin{zitemize}
    \item Developed a Python program to simulate and analyze spectroscopic lenses to place limits on Dark Energy Spectroscopic Instrument observation parameters.
    \item Code is available on GitHub and results were presented at the 237th meeting of the American Astronomical Society.
\end{zitemize}
\vspace*{5pt}

\subtext{{Electronics Research Intern}\hfill Dec 2021 - Feb 2022}
\begin{zitemize}
    \item Designed a circuit for an automatic hand sanitizer dispenser with MATLAB, Simulink, and Eagle CAD.
\end{zitemize}

\end{document}